
% LaTeX Beamer Presentation - Gestion de Projet
% Template: LaTeXTemplates.com (v2.0)
% Author: Abdelali EL BOUAZAOUI

\documentclass[11pt]{beamer}

\usetheme{Madrid}
\usecolortheme{dolphin}
\usefonttheme{default}
\useinnertheme{circles}
\setbeamercovered{transparent} % Pour les animations de transition

\usepackage[utf8]{inputenc}
\usepackage[french]{babel}
\usepackage{booktabs}
\usepackage{graphicx}
\usepackage{palatino}
\usepackage[default]{opensans}

\graphicspath{{Images/}{./}}

\title[Cours Gestion de Projet]{Cours : Gestion de projet}
\subtitle{TP pour le cours Innovation pédagogique}
\author[Abdelali EL BOUAZAOUI]{Abdelali EL BOUAZAOUI}
\institute[UC]{Université Hassan II \\ \smallskip \textit{a.elbouazaoui@gmail.com}}
\date[\today]{Doctorat en science de l'ingénieur \\ \today}

\logo{\includegraphics[height=0.9cm]{logo_hassan2.png}} % logo à placer dans le dossier Images

\begin{document}

\begin{frame}
  \titlepage
\end{frame}

\begin{frame}
  \frametitle{Sommaire du cours}
  \tableofcontents
\end{frame}

\section*{Objectifs d'apprentissage}
\begin{frame}
  \frametitle{Objectifs d'apprentissage}
  \pause
  \begin{itemize}
    \item Démarrer et valider un projet avec une charte solide \pause
    \item Maîtriser les outils de planification (WBS, Gantt, PERT) \pause
    \item Appliquer la triple contrainte pour optimiser un projet \pause
    \item Identifier et gérer les risques pour éviter les dérives \pause
    \item Capitaliser les apprentissages à la clôture \pause
    \item Impliquer efficacement les parties prenantes
  \end{itemize}
\end{frame}

% ------------------- Contenu par chapitres ------------------------

\section{Chapitre 1 : Introduction à la Gestion de Projet}
\begin{frame}{Fondamentaux de la gestion de projet}
  \pause
  \begin{itemize}
    \item Qu'est-ce qu'un projet ? \pause
    \item Différences entre projet et opérations courantes \pause
    \item Cycle de vie : initiation, planification, exécution, clôture \pause
    \item Compétences clés du chef de projet
  \end{itemize}
  \begin{figure}\centering
    \includegraphics[width=0.75\linewidth]{Images/project_lifecycle.png}
    \caption{Cycle de vie d'un projet}
  \end{figure}
\end{frame}

\begin{frame}{Étude de cas}
  \pause
  Pourquoi certains projets échouent dès le départ ? \pause
  \begin{itemize}
    \item Manque de clarté des objectifs \pause
    \item Ressources mal allouées \pause
    \item Communication insuffisante \pause
    \item Absence d'alignement stratégique
  \end{itemize}
\end{frame}

\section{Chapitre 2 : Cadrage et Validation}
\begin{frame}{Charte et faisabilité}
  \pause
  \begin{itemize}
    \item Identification des besoins \pause
    \item Analyse de faisabilité \pause
    \item Élaboration de la charte : objectifs, parties prenantes, contraintes \pause
    \item Validation par le sponsor
  \end{itemize}
\end{frame}

\begin{frame}{Atelier pratique}
  \pause
  Rédaction d’une charte pour un projet réel.\pause
  \begin{figure}\centering
    \includegraphics[width=0.6\linewidth]{Images/project_charter.png}
    \caption{Exemple de charte de projet}
  \end{figure}
\end{frame}

\section{Chapitre 3 : Planification – Périmètre et Livrables}
\begin{frame}{Définir les livrables du projet}
  \pause
  \begin{itemize}
    \item Recueil des besoins : interview, brainstorming, MoSCoW \pause
    \item Structuration des livrables avec la WBS \pause
    \item Gestion des exigences et des changements
  \end{itemize}
\end{frame}

\begin{frame}{Exercice pratique}
  \pause
  Créer un WBS pour un projet digital.
  \begin{figure}\centering
    \includegraphics[width=0.75\linewidth]{Images/wbs_example.png}
    \caption{Exemple de WBS}
  \end{figure}
\end{frame}

\section{Chapitre 4 : La Triple Contrainte}
\begin{frame}{Coût, Délai, Qualité}
  \pause
  \begin{itemize}
    \item Comprendre l’équilibre coût-délai-qualité \pause
    \item Estimation des coûts : analogique, paramétrique, bottom-up \pause
    \item Planification : diagrammes de Gantt, PERT, CPM
  \end{itemize}
\end{frame}

\begin{frame}{Simulation}
  \pause
  Optimiser un projet sous contraintes.\pause
  \begin{figure}\centering
    \includegraphics[width=0.7\linewidth]{Images/triple_contrainte.png}
    \caption{La triple contrainte en gestion de projet}
  \end{figure}
\end{frame}

\section{Chapitre 5 : Ressources et Parties Prenantes}
\begin{frame}{Acteurs et ressources projet}
  \pause
  \begin{itemize}
    \item Identification des parties prenantes : matrice pouvoir/intérêt \pause
    \item Gestion des rôles et des conflits \pause
    \item Communication et engagement des équipes
  \end{itemize}
\end{frame}

\begin{frame}{Étude de cas}
  \pause
  Gérer les attentes divergentes dans un projet.\pause
  \begin{figure}\centering
    \includegraphics[width=0.65\linewidth]{Images/stakeholder_matrix.png}
    \caption{Matrice parties prenantes}
  \end{figure}
\end{frame}

\section*{Références}
\begin{frame}{Ouvrages recommandés}
  \pause
  \begin{itemize}
    \item \textit{Le Guide PMBOK} – Project Management Institute (PMI) \pause
    \item \textit{Gestion de projet pour les Nuls} – Stanley E. Portny \pause
    \item \textit{Scrum} – Jeff Sutherland
  \end{itemize}
\end{frame}

\begin{frame}[plain]
  \begin{center}
    {\Huge Merci de votre attention !}\\[1em]
    {\LARGE Questions ?}
  \end{center}
\end{frame}

\end{document}
